\documentclass{article}\usepackage{graphicx, color}
%% maxwidth is the original width if it is less than linewidth
%% otherwise use linewidth (to make sure the graphics do not exceed the margin)
\makeatletter
\def\maxwidth{ %
  \ifdim\Gin@nat@width>\linewidth
    \linewidth
  \else
    \Gin@nat@width
  \fi
}
\makeatother

\definecolor{fgcolor}{rgb}{0.2, 0.2, 0.2}
\newcommand{\hlnumber}[1]{\textcolor[rgb]{0,0,0}{#1}}%
\newcommand{\hlfunctioncall}[1]{\textcolor[rgb]{0.501960784313725,0,0.329411764705882}{\textbf{#1}}}%
\newcommand{\hlstring}[1]{\textcolor[rgb]{0.6,0.6,1}{#1}}%
\newcommand{\hlkeyword}[1]{\textcolor[rgb]{0,0,0}{\textbf{#1}}}%
\newcommand{\hlargument}[1]{\textcolor[rgb]{0.690196078431373,0.250980392156863,0.0196078431372549}{#1}}%
\newcommand{\hlcomment}[1]{\textcolor[rgb]{0.180392156862745,0.6,0.341176470588235}{#1}}%
\newcommand{\hlroxygencomment}[1]{\textcolor[rgb]{0.43921568627451,0.47843137254902,0.701960784313725}{#1}}%
\newcommand{\hlformalargs}[1]{\textcolor[rgb]{0.690196078431373,0.250980392156863,0.0196078431372549}{#1}}%
\newcommand{\hleqformalargs}[1]{\textcolor[rgb]{0.690196078431373,0.250980392156863,0.0196078431372549}{#1}}%
\newcommand{\hlassignement}[1]{\textcolor[rgb]{0,0,0}{\textbf{#1}}}%
\newcommand{\hlpackage}[1]{\textcolor[rgb]{0.588235294117647,0.709803921568627,0.145098039215686}{#1}}%
\newcommand{\hlslot}[1]{\textit{#1}}%
\newcommand{\hlsymbol}[1]{\textcolor[rgb]{0,0,0}{#1}}%
\newcommand{\hlprompt}[1]{\textcolor[rgb]{0.2,0.2,0.2}{#1}}%

\usepackage{framed}
\makeatletter
\newenvironment{kframe}{%
 \def\at@end@of@kframe{}%
 \ifinner\ifhmode%
  \def\at@end@of@kframe{\end{minipage}}%
  \begin{minipage}{\columnwidth}%
 \fi\fi%
 \def\FrameCommand##1{\hskip\@totalleftmargin \hskip-\fboxsep
 \colorbox{shadecolor}{##1}\hskip-\fboxsep
     % There is no \\@totalrightmargin, so:
     \hskip-\linewidth \hskip-\@totalleftmargin \hskip\columnwidth}%
 \MakeFramed {\advance\hsize-\width
   \@totalleftmargin\z@ \linewidth\hsize
   \@setminipage}}%
 {\par\unskip\endMakeFramed%
 \at@end@of@kframe}
\makeatother

\definecolor{shadecolor}{rgb}{.97, .97, .97}
\definecolor{messagecolor}{rgb}{0, 0, 0}
\definecolor{warningcolor}{rgb}{1, 0, 1}
\definecolor{errorcolor}{rgb}{1, 0, 0}
\newenvironment{knitrout}{}{} % an empty environment to be redefined in TeX

\usepackage{alltt}
\usepackage{mathtools}
\IfFileExists{upquote.sty}{\usepackage{upquote}}{}
\begin{document}

\title{Finding an initial Point}

\author{Mike Flynn}
\maketitle
\section*{Minimizing distance from the origin}

The thought process behind this idea was so: We need to get a point
from the inside of a polytope, given the constraint equations. These
equations give us a k-plane that will always go through the plane:
$x_1 + x_2 + \dots + x_n = 1$. Since the ``triangle'' defined by the
intersection of this plane and the constraints $x_i > 0$ is already
very close to zero, I assumed that any intersection of this plane with
another constraint would automatically have it's closest point within
the ``triangle'', however, this is only true if the intersection
``goes through the triangle'' in the first place, or so it seems.
\\ \\
\noindent The derivation is as follows:
\\ \\
\noindent The contraint equation $Ax=b$ can be though of geometrically
as sequentially restricting $x$ to plane by plane, with each row of
$A$ being a plane. It is a basic result of linear algebra that any
plane can be represented by the equation $ n * x = c$ where n is a
vector that is normal to the plane, and c is some constant
(i.e. $x+y+z = [1,1,1] * [x,y,z] = 1$). This is exactly what the rows
of $A$ are, normal vectors.

$$ A = \begin{bmatrix}\mathbf{n_1} \\\mathbf{n_2} \\ \vdots \\ \mathbf{n_n} \end{bmatrix} $$
\noindent
This will help because the shortest distance between a point and a
plane is the perpendicular distance between them, and therefore this
distance vector must be in line with a normal vector. For this
distance from the origin, this vector must be the position. It follows
that:

$$ x = \sum_{j=0}^n \mathbf{n_j}c_j = A^T c$$

\noindent
We can now use the two equations for $x$ to solve for it:

$$ A x = b$$
$$ x = A^Tc$$
Therefore:

$$ c = (AA^T)^{-1}b $$
Finally:

$$ x = A^T(AA^T)^{-1}b$$

\noindent
I implemented this code as follows, which works fine in low dimensions
but stop's being so good in higher dimensions:

\begin{knitrout}
\definecolor{shadecolor}{rgb}{0.969, 0.969, 0.969}\color{fgcolor}\begin{kframe}
\begin{alltt}
find_x0 <- \hlfunctioncall{function}(A, b) \{
    \hlfunctioncall{return}(\hlfunctioncall{t}(A) %*% \hlfunctioncall{solve}(A %*% \hlfunctioncall{t}(A)) %*% b)
\}
A = \hlfunctioncall{matrix}(1, ncol = 3, nrow = 1)
A = \hlfunctioncall{rbind}(A, \hlfunctioncall{rnorm}(3))  \hlcomment{# A random constraint, mimicking value}
b = A %*% \hlfunctioncall{c}(0.2, 0.3, 0.5)
x0 = \hlfunctioncall{find_x0}(A, b)
b
\end{alltt}
\begin{verbatim}
##         [,1]
## [1,]  1.0000
## [2,] -0.4762
\end{verbatim}
\begin{alltt}
A %*% x0
\end{alltt}
\begin{verbatim}
##         [,1]
## [1,]  1.0000
## [2,] -0.4762
\end{verbatim}
\begin{alltt}
x0
\end{alltt}
\begin{verbatim}
##        [,1]
## [1,] 0.2243
## [2,] 0.4048
## [3,] 0.3709
\end{verbatim}
\begin{alltt}

\hlcomment{## This stops working nicely for jan}
\hlfunctioncall{data}(jan)
A = \hlfunctioncall{matrix}(1, ncol = \hlfunctioncall{nrow}(jan), nrow = 1)
A = \hlfunctioncall{rbind}(A, jan$value)
A = \hlfunctioncall{rbind}(A, jan$growth)
b = A %*% jan$portfolio
b
\end{alltt}
\begin{verbatim}
##         [,1]
## [1,]  1.0000
## [2,]  1.9916
## [3,] -0.2689
\end{verbatim}
\begin{alltt}
x0 = \hlfunctioncall{find_x0}(A, b)
\hlfunctioncall{length}(\hlfunctioncall{which}(x0 > 0))
\end{alltt}
\begin{verbatim}
## [1] 2113
\end{verbatim}
\begin{alltt}
\hlfunctioncall{length}(\hlfunctioncall{which}(x0 < 0))
\end{alltt}
\begin{verbatim}
## [1] 887
\end{verbatim}
\begin{alltt}
A %*% x0
\end{alltt}
\begin{verbatim}
##         [,1]
## [1,]  1.0000
## [2,]  1.9916
## [3,] -0.2689
\end{verbatim}
\end{kframe}
\end{knitrout}


As you can see, some of the weights are negative, which we do not
want.
\end{document}
