\documentclass{article}\usepackage{graphicx, color}
%% maxwidth is the original width if it is less than linewidth
%% otherwise use linewidth (to make sure the graphics do not exceed the margin)
\makeatletter
\def\maxwidth{ %
  \ifdim\Gin@nat@width>\linewidth
    \linewidth
  \else
    \Gin@nat@width
  \fi
}
\makeatother

\definecolor{fgcolor}{rgb}{0.2, 0.2, 0.2}
\newcommand{\hlnumber}[1]{\textcolor[rgb]{0,0,0}{#1}}%
\newcommand{\hlfunctioncall}[1]{\textcolor[rgb]{0.501960784313725,0,0.329411764705882}{\textbf{#1}}}%
\newcommand{\hlstring}[1]{\textcolor[rgb]{0.6,0.6,1}{#1}}%
\newcommand{\hlkeyword}[1]{\textcolor[rgb]{0,0,0}{\textbf{#1}}}%
\newcommand{\hlargument}[1]{\textcolor[rgb]{0.690196078431373,0.250980392156863,0.0196078431372549}{#1}}%
\newcommand{\hlcomment}[1]{\textcolor[rgb]{0.180392156862745,0.6,0.341176470588235}{#1}}%
\newcommand{\hlroxygencomment}[1]{\textcolor[rgb]{0.43921568627451,0.47843137254902,0.701960784313725}{#1}}%
\newcommand{\hlformalargs}[1]{\textcolor[rgb]{0.690196078431373,0.250980392156863,0.0196078431372549}{#1}}%
\newcommand{\hleqformalargs}[1]{\textcolor[rgb]{0.690196078431373,0.250980392156863,0.0196078431372549}{#1}}%
\newcommand{\hlassignement}[1]{\textcolor[rgb]{0,0,0}{\textbf{#1}}}%
\newcommand{\hlpackage}[1]{\textcolor[rgb]{0.588235294117647,0.709803921568627,0.145098039215686}{#1}}%
\newcommand{\hlslot}[1]{\textit{#1}}%
\newcommand{\hlsymbol}[1]{\textcolor[rgb]{0,0,0}{#1}}%
\newcommand{\hlprompt}[1]{\textcolor[rgb]{0.2,0.2,0.2}{#1}}%

\usepackage{framed}
\makeatletter
\newenvironment{kframe}{%
 \def\at@end@of@kframe{}%
 \ifinner\ifhmode%
  \def\at@end@of@kframe{\end{minipage}}%
  \begin{minipage}{\columnwidth}%
 \fi\fi%
 \def\FrameCommand##1{\hskip\@totalleftmargin \hskip-\fboxsep
 \colorbox{shadecolor}{##1}\hskip-\fboxsep
     % There is no \\@totalrightmargin, so:
     \hskip-\linewidth \hskip-\@totalleftmargin \hskip\columnwidth}%
 \MakeFramed {\advance\hsize-\width
   \@totalleftmargin\z@ \linewidth\hsize
   \@setminipage}}%
 {\par\unskip\endMakeFramed%
 \at@end@of@kframe}
\makeatother

\definecolor{shadecolor}{rgb}{.97, .97, .97}
\definecolor{messagecolor}{rgb}{0, 0, 0}
\definecolor{warningcolor}{rgb}{1, 0, 1}
\definecolor{errorcolor}{rgb}{1, 0, 0}
\newenvironment{knitrout}{}{} % an empty environment to be redefined in TeX

\usepackage{alltt}
\usepackage{amsmath}
\IfFileExists{upquote.sty}{\usepackage{upquote}}{}


\begin{document}

\title{Sampling from a Convex Polytope}
\author{Mike Flynn}
\maketitle

\section*{Hit-and-Run algorithm}
To sample from the hull of a convex polytope, a random-walk algorithm
is typically used. The ``hit-and-run'' type algorithm uses the
following steps:

\begin{enumerate}
  \item Start from an intial solution $x_0$
  \item Pick a random direction in the polytope, $u$
  \item Isolate the segment $s$ connecting $x_0$ to a wall of the polytope
    in that direction $u$
  \item Sample uniformly from the segment $s$

\end{enumerate}


\noindent In Detail:
\\ \\
\noindent The polytope is defined as the set: $\{x| Ax = A x_0, x > 0\}$
where $x$ is a vector of length $n$, A is a $m \times n$ matrix of
constraints and $x_0$ an original solution. This set is geometrically
an intersection of $m$ n-planes defined by the rows of $A$ and the
half spaces $x_i > 0$. An easy case to picture is the 1-row A:
$[1,1,1]$ with initial solution $(.3,.3,.4)$. This corresponds the the
plane $x + y + z = 1$. The intersection of this plane with $x,y,z>0$
leaves only the part in the first octant, a triangle.

Because we must at another solution to $Ax = Ax_0$ in the end, picking
a ``random'' direction will not be a random direction in the space of
$x$ but rather a random direction in the k-plane that is
$Ax=Ax_0$. This is done by finding an orthogonal basis of the null
space of $A$: $Z_1, Z_2,\dots,Z_k$, which will necessarily be
orthogonal vectors in the k-plane. These basis vectors are weighted
uniformly by sampling their weights from an exponential distribution
and dividing by their sum. Therefore: $$ u = \sum_{j=0}^kZ_jr_j$$
where $r_j$ is the normalized, exponentially distributed random weight.

We sample along the segment $s$ by saying that $x_{i+1} = x_i + t*u$
where $t$ is some scalar parameter, bounded by the limits of $s$. To
sample uniformly on $s$, we merely must sample uniformly on $t$,
bounded by the limits of $s$. To find the limits of $t$ we must
recognize that for each index $i$: $$ x_i + t*u_i > 0 $$ of which
there are only 2 important cases: when $u_i > 0$ and when $u_i <
0$. This is because when we solve for the limits of $t$ we simply
divide by $u_i$ to get 2 equations:

$$ t_i > -\frac{x_i}{u_i} \text{ for } u_i > 0 $$
$$ \text{and}$$
$$ t_i < -\frac{x_i}{u_i} \text{ for } u_i < 0 $$

Therefore the largest $t$ can be is large enough so that it is still
less than the smallest such right hand side for the second equation,
set by $x_i$, and likewise, must be greater than the largest right
hand side for the first equation. Formally:

$$  t_{\text{max}} = \text{Min}(-\frac{x_i}{u_i}) \text{ for } u_i < 0 $$
$$ \text{and}$$
$$  t_{\text{min}} = \text{Max}(-\frac{x_i}{u_i}) \text{ for } u_i > 0 $$

After these are figured out, $t$ can be drawn from a uniform
distribution between $t_{\text{min}}$ and $t_{\text{max}}$ to walk
randomly on the simplex.
\newpage

\noindent A demonstration:

\begin{knitrout}
\definecolor{shadecolor}{rgb}{0.969, 0.969, 0.969}\color{fgcolor}\begin{kframe}
\begin{alltt}
\hlfunctioncall{require}(MASS)
\hlfunctioncall{require}(scatterplot3d)
\hlcomment{#' Uniformly samples from \{A*x=A*x0\} U \{x>0\}}
getWeights.hnr <- \hlfunctioncall{function}(A, x0, n, discard) \{
    
    y = x0
\hlcomment{    ## resolve weird quirk in Null() function}
    \hlfunctioncall{if} (\hlfunctioncall{ncol}(A) == 1) \{
        Z = \hlfunctioncall{Null}(A)
    \} else \{
        Z = \hlfunctioncall{Null}(\hlfunctioncall{t}(A))
    \}
    X = \hlfunctioncall{matrix}(0, nrow = \hlfunctioncall{length}(x0), ncol = n + discard)
    \hlfunctioncall{for} (i in 1:(n + discard)) \{
\hlcomment{        ## u is a random unit vector}
        r = \hlfunctioncall{rexp}(\hlfunctioncall{ncol}(Z))
        r = r/\hlfunctioncall{sum}(r)
        
\hlcomment{        ## d is a unit vector in the appropriate k-plane pointing in a random}
\hlcomment{        ## direction}
        u = Z %*% r
        c = y/u
\hlcomment{        ## determine intersections of x + t*u with edges}
        tmin = \hlfunctioncall{max}(-c[u > 0])
        tmax = \hlfunctioncall{min}(-c[u < 0])
        
\hlcomment{        ## writeLines(paste('tmin: ', tmin, '\textbackslash{}ntmax: ', tmax, '\textbackslash{}n', sep = ''))}
\hlcomment{        ## chose a point on the line segment}
        y = y + (tmin + (tmax - tmin) * \hlfunctioncall{runif}(1)) * u
        X[, i] = y
    \}
    \hlfunctioncall{return}(X[, (discard + 1):\hlfunctioncall{ncol}(X)])
\}


\hlcomment{## Sample from the triangle \{x+y+z =1\} U \{x>0\}}
Amat = \hlfunctioncall{matrix}(\hlfunctioncall{c}(1, 1, 1), ncol = 3, nrow = 1)
x0 = \hlfunctioncall{c}(0.3, 0.2, 0.5)
w = \hlfunctioncall{getWeights.hnr}(Amat, x0, 1000, 5)

\hlfunctioncall{scatterplot3d}(x = w[1, ], y = w[2, ], z = w[3, ], angle = 160)
\end{alltt}
\end{kframe}
\includegraphics[width=\maxwidth]{figure/demohnr} 

\end{knitrout}

\newpage
\section{Finding $x_0$}
\subsection*{Minimizing distance from the origin}

The thought process behind this idea was so: We need to get a point
from the inside of a polytope, given the constraint equations. These
equations give us a k-plane that will always go through the plane:
$x_1 + x_2 + \dots + x_n = 1$. Since the ``triangle'' defined by the
intersection of this plane and the constraints $x_i > 0$ is already
very close to zero, I assumed that any intersection of this plane with
another constraint would automatically have it's closest point within
the ``triangle'', however, this is only true if the intersection
``goes through the triangle'' in the first place, or so it seems.
\\ \\
\noindent The derivation is as follows:
\\ \\
\noindent The contraint equation $Ax=b$ can be though of geometrically
as sequentially restricting $x$ to plane by plane, with each row of
$A$ being a plane. It is a basic result of linear algebra that any
plane can be represented by the equation $ n * x = c$ where n is a
vector that is normal to the plane, and c is some constant
(i.e. $x+y+z = [1,1,1] * [x,y,z] = 1$). This is exactly what the rows
of $A$ are, normal vectors.

$$ A = \begin{bmatrix}\mathbf{n_1} \\\mathbf{n_2} \\ \vdots \\ \mathbf{n_n} \end{bmatrix} $$
\noindent
This will help because the shortest distance between a point and a
plane is the perpendicular distance between them, and therefore this
distance vector must be in line with a normal vector. For this
distance from the origin, this vector must be the position. It follows
that:

$$ x = \sum_{j=0}^n \mathbf{n_j}c_j = A^T c$$

\noindent
We can now use the two equations for $x$ to solve for it:

$$ A x = b$$
$$ x = A^Tc$$
Therefore:

$$ c = (AA^T)^{-1}b $$
Finally:

$$ x = A^T(AA^T)^{-1}b$$

\noindent
I implemented this code as follows, which works fine in low dimensions
but stop's being so good in higher dimensions:

\begin{knitrout}
\definecolor{shadecolor}{rgb}{0.969, 0.969, 0.969}\color{fgcolor}\begin{kframe}
\begin{alltt}
find_x0 <- \hlfunctioncall{function}(A, b) \{
    \hlfunctioncall{return}(\hlfunctioncall{t}(A) %*% \hlfunctioncall{solve}(A %*% \hlfunctioncall{t}(A)) %*% b)
\}
A = \hlfunctioncall{matrix}(1, ncol = 3, nrow = 1)
A = \hlfunctioncall{rbind}(A, \hlfunctioncall{rnorm}(3))  \hlcomment{# A random constraint, mimicking value}
b = A %*% \hlfunctioncall{c}(0.2, 0.3, 0.5)
x0 = \hlfunctioncall{find_x0}(A, b)
b
\end{alltt}
\begin{verbatim}
##         [,1]
## [1,] 1.00000
## [2,] 0.03647
\end{verbatim}
\begin{alltt}
A %*% x0
\end{alltt}
\begin{verbatim}
##         [,1]
## [1,] 1.00000
## [2,] 0.03647
\end{verbatim}
\begin{alltt}
x0
\end{alltt}
\begin{verbatim}
##        [,1]
## [1,] 0.2418
## [2,] 0.4024
## [3,] 0.3558
\end{verbatim}
\begin{alltt}

\hlcomment{## This stops working nicely for jan}
\hlfunctioncall{data}(jan)
A = \hlfunctioncall{matrix}(1, ncol = \hlfunctioncall{nrow}(jan), nrow = 1)
A = \hlfunctioncall{rbind}(A, jan$value)
A = \hlfunctioncall{rbind}(A, jan$growth)
b = A %*% jan$portfolio
b
\end{alltt}
\begin{verbatim}
##         [,1]
## [1,]  1.0000
## [2,]  1.9916
## [3,] -0.2689
\end{verbatim}
\begin{alltt}
x0 = \hlfunctioncall{find_x0}(A, b)
\hlfunctioncall{length}(\hlfunctioncall{which}(x0 > 0))
\end{alltt}
\begin{verbatim}
## [1] 2113
\end{verbatim}
\begin{alltt}
\hlfunctioncall{length}(\hlfunctioncall{which}(x0 < 0))
\end{alltt}
\begin{verbatim}
## [1] 887
\end{verbatim}
\begin{alltt}
A %*% x0
\end{alltt}
\begin{verbatim}
##         [,1]
## [1,]  1.0000
## [2,]  1.9916
## [3,] -0.2689
\end{verbatim}
\end{kframe}
\end{knitrout}


As you can see, some of the weights are negative, which we do not
want.

\newpage
\section{Changing up: going to quadratic programming}

After suggestions from colleagues, we move back to an attempt at
quadratic programming to solve our problem. The package
\verb+quadprog+ will solve the following problem for $b$:

$$ \text{Min}(-d^Tb + 1/2b^TDb) \text{ given that } A^Tb >= b_0$$

Since it similar to linear programming, it can be assumed that this
method will bias towards the edges, which in fact it does, however, it
does not exclusively go to corner solutions, like linear programming
does, and so might be a better way to go about solving this. An
example in 3D:

\begin{knitrout}
\definecolor{shadecolor}{rgb}{0.969, 0.969, 0.969}\color{fgcolor}\begin{kframe}
\begin{alltt}
getWeights.quad <- \hlfunctioncall{function}(A, b, n, verbose = FALSE) \{
    
    w = \hlfunctioncall{matrix}(0, ncol = n, nrow = \hlfunctioncall{ncol}(A))
    \hlfunctioncall{if} (verbose) 
        \hlfunctioncall{cat}(\hlstring{"Portfolios created: 0"})
    \hlfunctioncall{for} (i in 1:n) \{
        sol = \hlfunctioncall{solve.QP}(Dmat = \hlfunctioncall{diag}(2, \hlfunctioncall{ncol}(A)), dvec = \hlfunctioncall{rnorm}(\hlfunctioncall{ncol}(A)), Amat = \hlfunctioncall{t}(\hlfunctioncall{rbind}(A, 
            \hlfunctioncall{diag}(1, \hlfunctioncall{ncol}(A)))), bvec = \hlfunctioncall{c}(b, \hlfunctioncall{rep}(0, \hlfunctioncall{ncol}(A))), meq = \hlfunctioncall{nrow}(A))
        w[, i] = sol$solution
        \hlfunctioncall{if} (verbose) \{
            \hlfunctioncall{for} (j in 1:\hlfunctioncall{nchar}(\hlfunctioncall{paste}(i - 1))) \hlfunctioncall{cat}(\hlstring{"\textbackslash{}b"})
            \hlfunctioncall{cat}(\hlfunctioncall{paste}(i))
        \}
    \}
    \hlfunctioncall{return}(w)
\}

A = \hlfunctioncall{matrix}(1, ncol = 3, nrow = 1)
b = 1
w = \hlfunctioncall{getWeights.quad}(A, b, 1000)
\hlfunctioncall{scatterplot3d}(x = w[1, ], y = w[2, ], z = w[3, ], angle = 160)
\end{alltt}
\end{kframe}
\includegraphics[width=\maxwidth]{figure/3dexample} 

\end{knitrout}


As you can see we get many samples from the edges, but not all of
them. Turning the dimensions up to 4 and looking at a 3d cross-section:

\begin{knitrout}
\definecolor{shadecolor}{rgb}{0.969, 0.969, 0.969}\color{fgcolor}\begin{kframe}
\begin{alltt}
A = \hlfunctioncall{matrix}(1, ncol = 4, nrow = 1)
b = 1
w = \hlfunctioncall{getWeights.quad}(A, b, 1000)
\hlfunctioncall{scatterplot3d}(x = w[1, ], y = w[2, ], z = w[3, ], angle = 160)
\end{alltt}
\end{kframe}
\includegraphics[width=\maxwidth]{figure/4dexample} 

\end{knitrout}


This is an interesting pyramid shape and despite its bias towards the
edges, it seems like something that we would want. Unfortunately,
trying it out on \verb+jan+ does not seem to be a productive use of
time, as it takes quite a while.
\end{document}
